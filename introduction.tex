\chapter{Introduction} \label{ch:intro}
A \acf{qr} is a type of \acf{uav} that has received an increasing amount of attention recently with many applications being actively investigated. Possible applications include search and rescue, surveillance, reliable supply of food and medicines as disaster relief and object manipulation in construction and transportation. It has already proven itself useful for many tasks like multi-agent missions, mapping, explorations, transportation and entertainment such as acrobatic performances.

The inspiration for this research is build upon the idea of creating a system of multiple autonomous \a{qr}s for a cooperative towing task. The advantage of such a system for object manipulation is the possibility to reduce complexity of the individual robot, decrease cost over traditional robotic systems and high reliability. One can think of examples in nature, where individuals coordinate, cooperate and collaborate to perform tasks that they individually can not accomplish. Redundancy makes development of fail safe control methods possible and can extend the capabilities of a single robot. 						

***************************************\\
Bart: suggestion: creating a system of multiple autonomous QRs?\\
Nam: Zin aangepast.

***************************************\\

Considering a multi-agent task, one can think of multiple \a{qr}s assisting in the transportation of a common load. This cooperation can be executed in many ways, but this research focuses on \a{qr}s with a cable-suspended load in motion. The suspended object naturally continues to swing at the end of every movement. In case a residual motion can result in damage or in order to avoid obstacles and path following, an accurate positioning is required. Reducing the oscillation, or controlling the position of the suspended load might be necessary, but is challenging in the fact that this cable-suspended system is under-actuated. Possible objectives are minimizing the oscillations of the load during or after motion, minimizing the time to position the load, trajectory tracking, trajectory generation and obstacle avoidance.

***************************************\\
Bart: Suggestion place under motivation of research inspiration, this forms a nice bridge.\\
Nam: Ik heb dit juist naar boven geplaatst. The inspiration ..... single robot. 

***************************************\\
\newpage
\section{Aim and Motivation}\label{sec:int.motivation}
The aim is to control the position of a suspended load using a \a{qr}. Before considering multiple \a{qr}s, it is important to investigate the possibilities of a single \a{qr} system. Hence, in this research a single \a{qr} is considered for the transportation of a cable suspended load, which will exert additional forces and torques on the \a{qr}. This is a challenging control problem in the fact that the \a{qr} system is under-actuated. Adding a suspended load will add extra \a{DOF}s and oscillations of the load occur at the end of every movement. 

***************************************\\
Bart: Suggestion: Since QRs with suspended load are scarcely investigated it is important to investigate a single QR first before the multiple QRs can be considered.\\
Nam: Zie boven

***************************************\\
The system can be divided into two subsystems. The first subsystem is where the cable tension is non-zero and the distance between the \a{qr} and the load is defined by the cable length. Both \a{qr} and load are coupled as one system. The second subsystem is where the cable tension is zero, such that the \a{qr} and load in free fall are two separate decoupled systems. This research focuses on the first subsystem, such that the cable tension is non-zero. In order to control both subsystems, hybrid control must be applied, which is considered out of the scope of this research.

Former work on \a{qr} attitude- and position control often rely on linear control methods such as \acs{pid}\cite{bibid}, \a{mpc} \cite{Bangura2014} and \a{lqr} control \cite{bibid}. The dynamics are linearized around an equilibrium point, describing the system dynamics by a set of linear differential equations. 
%ADD references to pid and lqr controllers qr attitude
The control of a \a{qr}-Load system is a very specific case, and former work includes \a{mpc} \cite{PraveenThesis} and \a{lqr}\cite{bibid} control approaches, where an optimal control strategy is used to minimize the swing of the load. 
%ADD what more former work?

The reason that linear control near an equilibrium state is commonly applied, is partly to avoid difficulties that come with modeling and controlling the non-linearities of the system. Non-linear control systems are often governed by nonlinear differential equations and are able to represent the dynamics in a more realistic manner. However, linear control limits the system to small angle movements, as the optimization will not allow large angles that deviate to far from the linearized point. For applications that require fast aggressive maneuvers, this type of modeling and control will not be sufficient.

A possible nonlinear control approach is \acs{nmpc}, which is a variant of \a{mpc} that uses a nonlinear dynamical system to predict the required inputs. While these problems are convex in linear \a{mpc}, in nonlinear \a{mpc} they are not convex anymore. This poses challenges for both \a{nmpc} stability theory and numerical solution.
%ADD what more former work?

%ADD Reason to consider Geometric Control 
Nonlinear Geometric Control is a nonlinear model based control technique based on a modeling approach involving the concepts of differential geometry. This results in a globally defined coordinate-free dynamical model, while preventing issues regarding singularities, and enabling the design of controllers that offer almost-global convergence properties.

Former work includes a nonlinear geometric control of a \a{qr} \cite{Lee2010,Goodarzi2013a} and nonlinear geometric control of the load position, load attitude and \a{qr} attitude of a \a{qr}-Load system \cite{Sreenath2013a,Sreenath2013b,Tang2014}.
Geometric Control for the control of \a{qr} systems is used less frequently, despite the advantageous properties of differential geometry. 

%ADD MOTIVATION
This motivates to investigate the possibilities and limitations of the less used nonlinear Geometric Control design, and compare this to a commonly used linear control strategy.
The question is which advantages or disadvantages this nonlinear approach has compared to a linear approach, in terms of stability and performance.

%A controller is designed a priori to guarantee global asymptotic stability. 
%Unlike a \a{mpc} approach, the 

%ADD Contributions in this thesis

Different aspects involving the modeling and control for the QR-Load system must be investigated, for it can be expected that the non-linearity will have a great influence in the representation of the dynamics and the stability, accuracy and type of control design.

System consists of two sub-systems
Limited to subsystem where the tension of the cable is non-zero. 

\section{Organization of the Report}

%\begin{itemize}
%\item Chapter \ref{ch:intro}\\
In this first chapter, a brief introduction of the subject is given and the problem is described. This is followed by discussing the aim, motivation and contributions of this thesis for this research. The organization of the report is as follows.
%\item Chapter \ref{ch:model}\\

In Chapter \ref{ch:model} the dynamics of the \a{qr}-Load system is described by the laws of kinematics and the application of Newton's laws or Lagrangian mechanics. Opposed to the classical modeling techniques, it is also possible to describe the system's configuration space as a differentiable manifold using the tools of differential geometry. Geometric Mechanics is used to understand and derive the equations of motion of a system in order to allow its analysis and design. 

%\item Chapter \ref{ch:control}\\
The system dynamics are represented on nonlinear manifolds and this allows nonlinear geometric controllers to be designed on these same manifolds. The control design is presented in Chapter \ref{ch:control}. The controller has a cascaded structure, where the different control loops are accountable for different flight modes. 

Different tracking objectives can be defined in order to test the performance of both an \a{lqr} control design and a nonlinear Geometric Control design. 
Chapter \ref{ch:results} describes the experiments that are done to investigate the abilities and performance of a nonlinear Geometric Control design. 
The results and findings are presented and discussed.

%\item Chapter \ref{ch:conclusion}\\
In the final chapter a summary of the thesis is given, followed by the conclusions that were made based on the results of the experiments.
%CHECK
%and further conclusions that follow from the whole research 
Finally, recommendations are given which could serve as an starting point for future work. 

%\end{itemize}
