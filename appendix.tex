\chapter{Appendix}

\section{Derivation of Equations of motion}
\subsection{Load Dynamics}\label{sec.app:loaddyn}
%PROOF prop.3 Sreenath2013a. Also Sreenath2013b?
%DEFINE e_3 / R / f 

Let \lsymb{$ x_{CM} $}{Position \lsymb{$ CM $}{Center of Mass} of \a{qr}-Load system} denote the position of the center of mass of the combined Quadrotor-Load system, expressed in \IF. Which can be found by
\begin{align}\label{eq:CM}
\begin{split}
m_Q(x_Q-x_{CM})+m_L(x_L-x_{CM})&=0\\
(m_Q+m_L)x_{CM}&=m_Qx_Q+m_Lx_L
\end{split}
\end{align}
Applying the laws of motion to (\ref{eq:CM}) and inserting (\ref{eq:mod.xQ2xL}) gives the 
\begin{align}\label{key}
\begin{split}
(m_Q+m_L)\ddot{x}_{CM}&=fRe_3 - (m_Q+m_L)ge_3\\
%&=fRe_3 - (m_Q+m_L)ge_3\\
%(m_Q+m_L)\ddot{x}_{CM}&=m_Q\ddot{x}_Q + m_L\ddot{x}_L\\
%\\
%m_Q\ddot{x}_Q+m_L\ddot{x}_L&=fRe_3 - (m_Q+m_L)ge_3\\
%m_Q(\ddot{x}_L-L\ddot{q})+m_L\ddot{x}_L&=fRe_3 - (m_Q+m_L)ge_3\\
(m_Q+m_L)(\ddot{x}_L+ge_3)&= fRe_3+m_QL\ddot{q}
\end{split}
\end{align}

%ADD derivation of ddq (TANG2014)


\section{Derivation of LQR controller}\label{app:lqr}

\subsection{Modeling}
From Newton's laws follows
\begin{align}
x_Q&=fRe_3-m_Qge_3-Tq\\
x_L&=-m_Lge_3+Tq
\end{align}

$ x_Q $ and $ x_L $ are related by
\begin{equation}\label{key}
x_L = x_Q+Lq
\end{equation}

\begin{equation}\label{eq:app.QRpos}
\begin{aligned}
%ADD Not done yet
\ddot{x}&=\\
\ddot{y}&=\\
\ddot{z}&=
\end{aligned}
\end{equation}

\begin{equation}\label{eq:app.QRatt}
\begin{aligned}
%ADD Not done yet
\ddot{\phi}&=\\
\ddot{\theta}&=\\
\ddot{\psi}&=
\end{aligned}
\end{equation}

\begin{equation}\label{eq:app.Latt}
\begin{aligned}
%ADD Not done yet
\ddot{\phi}_L&=\\
\ddot{\theta}_L&=
\end{aligned}
\end{equation}





%	    ***************************************\\
%	    \begin{equation}
%	    ax = 
%	    \begin{bmatrix}
%	    (f*m*sin(\phi_q)*sin(\psi_q) + f*m_l*sin(\phi_q)*sin(\psi_q) - f*m_l*cos(\theta)^2*sin(\phi_q)*sin(\psi_q) + f*m*cos(\phi_q)*cos(\psi_q)*sin(\theta_q) + f*m_l*cos(\phi_q)*cos(\psi_q)*sin(\theta_q) + l*m*m_l*v_\theta^2*cos(\theta)*sin(\phi) + f*m_l*cos(\phi)^2*cos(\theta)^2*sin(\phi_q)*sin(\psi_q) + l*m*m_l*v_\phi^2*cos(\theta)^3*sin(\phi) - f*m_l*cos(\phi_q)*cos(\psi_q)*cos(\theta)^2*sin(\theta_q) + f*m_l*cos(\phi)^2*cos(\phi_q)*cos(\psi_q)*cos(\theta)^2*sin(\theta_q) + f*m_l*cos(\psi_q)*cos(\theta)*sin(\phi)*sin(\phi_q)*sin(\theta) + f*m_l*cos(\phi)*cos(\phi_q)*cos(\theta_q)*cos(\theta)^2*sin(\phi) - f*m_l*cos(\phi_q)*cos(\theta)*sin(\phi)*sin(\psi_q)*sin(\theta_q)*sin(\theta))/(m*(m + m_l))\\
%	    (l*m*m_l*v_\theta^2*sin(\theta) - f*m_l*cos(\psi_q)*cos(\theta)^2*sin(\phi_q) - f*m*cos(\psi_q)*sin(\phi_q) + f*m*cos(\phi_q)*sin(\psi_q)*sin(\theta_q) + f*m_l*cos(\phi_q)*cos(\theta)^2*sin(\psi_q)*sin(\theta_q) + l*m*m_l*v_\phi^2*cos(\theta)^2*sin(\theta) + f*m_l*cos(\phi)*cos(\phi_q)*cos(\theta_q)*cos(\theta)*sin(\theta) - f*m_l*cos(\theta)*sin(\phi)*sin(\phi_q)*sin(\psi_q)*sin(\theta) - f*m_l*cos(\phi_q)*cos(\psi_q)*cos(\theta)*sin(\phi)*sin(\theta_q)*sin(\theta))/(m*(m + m_l))\\
%	    -(g*m^2 + g*m*m_l - f*m*cos(\phi_q)*cos(\theta_q) - f*m_l*cos(\phi_q)*cos(\theta_q) + l*m*m_l*v_\theta^2*cos(\phi)*cos(\theta) + f*m_l*cos(\phi)^2*cos(\phi_q)*cos(\theta_q)*cos(\theta)^2 + l*m*m_l*v_\phi^2*cos(\phi)*cos(\theta)^3 + f*m_l*cos(\phi)*cos(\psi_q)*cos(\theta)*sin(\phi_q)*sin(\theta) - f*m_l*cos(\phi)*cos(\theta)^2*sin(\phi)*sin(\phi_q)*sin(\psi_q) - f*m_l*cos(\phi)*cos(\phi_q)*cos(\theta)*sin(\psi_q)*sin(\theta_q)*sin(\theta) - f*m_l*cos(\phi)*cos(\phi_q)*cos(\psi_q)*cos(\theta)^2*sin(\phi)*sin(\theta_q))/(m*(m + m_l))\\
%	    (- l*m*cos(\theta)*sin(\theta)*v_\phi^2 + f*cos(\psi_q)*cos(\theta)*sin(\phi_q) - f*cos(\phi)*cos(\phi_q)*cos(\theta_q)*sin(\theta) - f*cos(\phi_q)*cos(\theta)*sin(\psi_q)*sin(\theta_q) + f*sin(\phi)*sin(\phi_q)*sin(\psi_q)*sin(\theta) + f*cos(\phi_q)*cos(\psi_q)*sin(\phi)*sin(\theta_q)*sin(\theta))/(l*m)\\
%	    -(f*cos(\phi_q)*cos(\theta_q)*sin(\phi) + f*cos(\phi)*sin(\phi_q)*sin(\psi_q) - 2*l*m*v_\phi*v_\theta*sin(\theta) + f*cos(\phi)*cos(\phi_q)*cos(\psi_q)*sin(\theta_q))/(l*m*cos(theta))
%	    \end{bmatrix}
%	    \end{equation}	    
%	    
%	    ***************************************\\	

\subsection{Controller}
\begin{align}\label{eq:state}
\textbf{x}&=\begin{bmatrix}
\textbf{q}\\
\mathbf{\dot{q}}
\end{bmatrix}\\
\mathbf{q}&=\begin{bmatrix}
x&y&z&\phi&\theta&\psi&\theta_L&\psi_L
\end{bmatrix}^T\\
\mathbf{\dot{q}}&=\begin{bmatrix}
\dot{x}&\dot{y}&\dot{z}&\dot{\phi}&\dot{\theta}&\dot{\psi}&\dot{\theta}_L&\dot{\psi}_L
\end{bmatrix}^T\\
u&=\begin{bmatrix}
f&M_\phi&M_\theta&M_\psi
\end{bmatrix}^T
\end{align}

\begin{align}\label{eq:ss}
\mathbf{\dot{x} }&=A\mathbf{x}+Bu\\
y&=C\mathbf{x}+Du
\end{align}

%CHECK if needed?
\begin{equation}\label{key}
\mathbf{\dot{x}}=\textbf{f}(\mathbf{x,u})
\end{equation}

%ADD reference
By linearizing Equations \ref{eq:app.QRpos},\ref{eq:app.QRatt},\ref{eq:app.Latt} follows
\begin{align}\label{key}
%CHECK this equations
\ddot{x}&=-\frac{f}{m}\\
\ddot{y}&=\\
\ddot{z}&=
\end{align}

The mathematical model is linearized around the following operating points
\begin{align}\label{key}
\bar{\mathbf{x}}&=\begin{bmatrix}\bar{x}&\bar{y}&\bar{z}&\textbf{0}_{1\times13}\end{bmatrix}^T\\
\bar{\textbf{u}}&=\begin{bmatrix}
(m_Q+m_L)g&0&0&0
\end{bmatrix}^T
\end{align}

%CHECK are these asssumptions correct?
Assuming small angles, the following holds
\begin{align}\label{key}
\text{for } \gamma &= \phi, \theta, \psi, \theta_L, \psi_L\\
sin(\gamma)&\simeq \gamma\\
cos(\gamma)&\simeq 1\\
\dot{\gamma} &\simeq 0\\
F &\simeq (m_Q+m_L)g
\end{align}

\begin{equation}\label{key}
A=\frac{\partial \textbf{f}(x,u)}{\partial x}\mid _{	x=\bar{x},u=\bar{u}	}
\end{equation}
\begin{equation}\label{key}
B=\frac{\partial \textbf{f}(x,u)}{\partial u}\mid _{	x=\bar{x},u=\bar{u}	}
\end{equation}

\begin{equation}\label{key}
u=-K\left[\textbf{x}_{des}(t)-\textbf{x}(t)\right] 
\end{equation}


\section{Figures}
\begin{figure}[h!]
	\centering
	\makebox[\textwidth][c]{\includegraphics[width=.45\textwidth]{./StyleStuff/dcsc.png}}
	\caption{Simulink Command Filter\label{fig:app.CF}}
\end{figure}		

\section{\texttt{MATLAB} code}
\begin{equation}\label{key}
Test equation
\end{equation}
\subsection{A \matlab Listing}

\lstset{language=matlab}
\lstinputlisting{test.m}

%    \subsection{An appendix subsection with C++ Listing}
%
%    \lstset{language=C++}
%    \lstinputlisting{test.c}    

%    \chapter{Appendix: Figures}
%
%    \section{Test section (again?)}
%
%    Ok, all is well.