\nonumchap{Abstract}

A Quadrotor is a type of Unmanned Aerial Vehicle that has received an increasing amount of attention recently with many applications including search and rescue, surveillance, supply of food and medicines as disaster relief and object manipulation in construction and transportation.
An interesting subproblem of transportation, is the control of the position of a cable suspended load. The challenge is in the fact that the Quadrotor-Load system is highly nonlinear and under-actuated. The load cannot be controlled directly and has a natural swing at the end of each Quadrotor movement. 

This thesis presents a Nonlinear Geometric Control approach for the position tracking of a cable suspended load. In this work the dynamical model of the Quadrotor-Load system is limited to the subsystem where the cable tension is non-zero, which is analogous to modeling a rigid link between the Quadrotor and Load.
Based on the geometric properties of the system, Geometric Mechanics apply differential geometric techniques to systems modeling and control.
The Quadrotor-Load system dynamics are modeled on so called manifolds, smooth nonlinear geometric configuration spaces. 
Analyzing these geometric structures with the principles of differential geometry allows modeling in an unambiguous coordinate-free dynamic fashion, while avoiding the problem of singularities that would occur on local charts. Tracking error functions are defined on these same spaces, making it possible to design almost globally defined controllers.
%with the principles of geometric control making control design on these same spaces 

The main goal of this thesis is to investigate the possibilities and limitations of Nonlinear Geometric Control by evaluating the performance of tracking different load trajectories. 
A backstepping approach is applied to generate a cascaded structure with multiple nonlinear Geometric controllers, allowing control of several flight modes that are responsible for the control of 1) Quadrotor attitude, 2) Load attitude and 3) Load position. 
A Linear Quadratic Regulator is derived to compare control performance. Simulations are presented and the results of both the linear and nonlinear controller are discussed. 



%%CHANGE meer power / eye cather, wat gebeurt er in deze thesis
%Different cases are tested to investigate the possibilities and limitations of Nonlinear Geometric Control. 
%%ADD Results show
%Results show that....
%
%%BART
%***************************************\\
%Bart: Objective little bit more explicit, suggestion: The main goal of this thesis is to investigate the possibilities and limitations of Nonlinear Geometric Control by evaluating the performance of tracking different trajectories. To compare control performance, a linear Quadratic Regulator is derived. 
%
%At least it should be an eyecatcher for the reader to know what the objective of the thesis is
%
%***************************************\\

%
%***************************************\\
%Intro about GC.. 
%Reasons to consider GC..
%
%***************************************\\
%Where simple linear control methods are restricted to small angle movements, nonlinear control methods allow more aggressive and faster movements.  The goal of the project is to investigate the effects on load position tracking performance when the system is modeled and controlled via a Nonlinear Geometric Control approach.
%
%
%***************************************\\
%The Quadrotor-Load system is modeled in a compact and coordinate-free fashion which allows the inherent geometric properties of the system to be controlled. 
%%BOVENSTAAND IS VAAG?

%
%The main goal of this thesis is to research the effects on a cable-suspended load transportation using quadrotors, by involving complex or aggressive maneuvering through implementation of Non-Linear Geometric Control.
%Where linear control methods are restricted to small angle movement, non-linear control methods allow more aggressive movements. 

%
%		The main goal of this literature study is to review researches that have been done regarding the possibilities of manipulation and transportation of cable-suspended loads using quadrotors, possibly involving complex or aggressive maneuvering. Where linear control methods are restricted to small angle movement, non-linear control methods allow more aggressive movements, but are subject to complex 
%		***************************************\\
%		What is complex? If that results in super easy feedback loop, why not consider it?
%		
%		***************************************\\
%		mathematics and are more computationally intensive and less robust. 


%***************************************\\
%In the considered research papers, different modeling methods and control techniques are applied which are suitable for various specific applications. The models, derivation and underlying assumptions are studied and explained in detail. 		
%
%Super general. Can be placed in every paper.
%
%***************************************\\

%Furthermore, the studied control techniques are explained and their advantages are addressed. Several trajectory generation approaches and the related optimization techniques are studied. Their applications, with different purposes such as obstacle avoidance, time-optimal and swing-free trajectory planning are explained. The survey is concluded with a discussion about finding a suitable
%***************************************\\
%Define suitable
%
%***************************************\\
%control design to achieve the quadrotor-assisted task involving manipulation of a cable-suspended load.
%
%***************************************\\


		
\cleardoublepage