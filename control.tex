\chapter{Control Design} \label{ch:control}
%ADD intro: in this section etc
Section \ref{sec:con.nlgc} introduces Nonlinear Geometric Control and the control design structure is discussed. Due to under-actuation, the load position tracking problem requires a backstepping control approach in order to control different flight modes.\\
In this section the tracking errors are defined on nonlinear manifolds, similar to the representation of the system dynamics.

%For the control of the different flight modes, the controllers are designed on the nonlinear geometric space, 
%being functions of the earlier described tracking errors. 
The controllers that are designed for the control of the different flight modes, are discussed in Sections \ref{sec:con.qratt}, \ref{sec:con.loadatt} and \ref{sec:con.loadpos}. 

%ADD 
%deze references introduceren
%\cite{Bullo2005,Jurdjevic1997}

\section{Nonlinear Geometric Control}\label{sec:con.nlgc}
Many control systems are developed for the standard form of ordinary differential equations, namely $ \dot{x}=f(x,u) $, where the state and the control input are denoted by $ x $ and $ u $. It is assumed that the state and the control input lie in Euclidean spaces, and the system equations are defined in terms of smooth functions between Euclidean spaces. However, for many interesting mechanical systems, the configuration space cannot be expressed globally as a Euclidean space.

Geometric Control Theory is the study of how the geometry of the state space influences controls problems. 
In control systems engineering, the underlying geometric features of a dynamic system are often not considered carefully. 
Differential geometric control techniques utilize the geometric properties for control system design and analysis.
The objective is to describe the system dynamics and control inputs on manifolds instead of local charts. In contrast to locally defined linear control, nonlinear geometric control can be defined almost globally, avoiding singularities that occur in the representation of large angles and complex maneuvering.

Attitude control systems naturally evolve on non-linear configurations such as $ \mathbb{S}^2 $ and $ SO(3) $. 
Tracking control system can be developed on $ SO(3) $, therefore it avoids singularities of Euler-Angles.

Global nonlinear dynamics of various classes of closed loop attitude control systems have been studied in recent years \cite{Chaturvedi2011a}.
In contrast to hybrid control systems \cite{Gillula2010}, complicated reachability set analysis is not required to guarantee safe switching between different flight modes, as the region of attraction for each flight mode covers the configuration space almost globally.

\paragraph{Backstepping Control}\label{sec:con.back}
A backstepping approach, or cascade control, is a Lyapunov based technique to design the control of nonlinear dynamical systems and ensuring Lyapunov stability. This approach is commonly used for the control of \a{qr}s \cite{Mahony2012} and will also be used in this research for the control of the load trajectory tracking problem.
The principle is to create a cascaded structure by starting with a stable system as a base, then "stepping back" from this base to add a control loop around it that stabilizes the added subsystem to enable the control of another state. This is repeated until the final external control is reached.
The control law is designed by using states as virtual control signals. Each loop computes a virtual command signal, denoted by subscript $ c $, as a tracking command for the inner loops. 

%ADD 
%Justify choice of parameters. Up to what level can we push the system? Where can we find more info about domain of attractions.
%How to choose parameters and how to select gains for errors

% CHECK nodig?
%\url{http://www.control.lth.se/media/Education/EngineeringProgram/FRTN05/2013/lec09_2013eight.pdf}
%We want to design a state feedback u=u(x) that stabilizes
%\begin{equation}\label{key}
%\begin{aligned}
%\dot{x}_1&=f(x_1)+g(x_1)x_2\\
%\dot{x}_2&=u
%\end{aligned}
%\end{equation}
%Idea is to see system as a cascade connection. Design controller first for inner loop, then for the outer.
%
%Back-Stepping Lemma
%\textbf{Lemma}: Let $ z=\begin{pmatrix}x_1,\cdots,x_{k-1}\end{pmatrix}^T$ and 
%\begin{equation}\label{key}
%\begin{aligned}
%\dot{z}&=f(z)+g(z)x_k\\
%\dot{x}_k&=u
%\end{aligned}
%\end{equation}
%Assume $ \phi(0)=0 $, $ f(0)=0 $,
%\begin{equation}\label{key}
%\dot{z}=f(z)+g(z)\phi(z)
%\end{equation}
%stable, and $ V(z) $ a Lyapunov function (with $ \dot{V}\leq-W $). Then, 
%\begin{equation}\label{key}
%u=\frac{d\phi}{dz}\begin{pmatrix}
%f(z)+g(z)x_k
%\end{pmatrix}-\frac{dV}{dz}g(z)-(x_k-\phi(z))
%\end{equation}
%stabilizes $ x=0 $ with $ V(z)+(x_k-\phi(z))^2/2 $ begin a Lyapunov function.
%
%The backstepping approach determines how to stabilize the {\displaystyle \mathbf {x} } \mathbf {x}  subsystem using {\displaystyle z_{1}} z_{1}, and then proceeds with determining how to make the next state {\displaystyle z_{2}} z_{2} drive {\displaystyle z_{1}} z_{1} to the control required to stabilize {\displaystyle \mathbf {x} } \mathbf {x} . Hence, the process "steps backward" from {\displaystyle \mathbf {x} } \mathbf {x}  out of the strict-feedback form system until the ultimate control {\displaystyle u} u is designed.
%***************************************\\

Because the \a{qr} has only four actuators, it is not possible to control all \a{DOF}s of the \a{qr}-Load system simultaneously. The backstepping approach allows control of different flight modes in which parts of the \a{DOF}s are controlled. The flight modes and their functions are defined as follows
\begin{outline}
\1 QR Attitude Controlled Mode 
\2 Track a desired QR attitude $ R_d(t) $ and a heading direction $ b_{1_d}(t) $
\2 Give a desired input $ M $ for system
\1 Load Attitude Controlled Mode 
\2 Track a desired load attitude command $ q_d(t) $
\2 Give a computed \a{qr} attitude $ R_c $ for the \a{qr} attitude controller (instead of $ R_d(t) $)
\1 Load Position Controlled Mode
\2 Track a desired load position $ x_{L,d}(t) $
\2 Give a computed load attitude $ q_c $ for the load attitude controller (instead of $qR_d(t) $)
\end{outline}
where the subscript $ \cdot_d $ denotes a desired tracking reference, and $ \cdot_c $ denotes a computed value that is calculated as a tracking reference. 

The controller that is used in this research is shown in Figure \ref{fig:con.loop}. The lowest levels have the highest bandwidth and are in control of the rotor rotational speeds $ \omega_i $, the total force $ f $ and moments $ M $. The next level controls the load attitude $ q $, and the top level controls the load position $ x_L $. 
\begin{figure}[h!]
	\centering
	\makebox[\textwidth][c]{\includegraphics[trim={0 0 0 9cm},clip,width=.95\textwidth]{./StyleStuff/backstepQR2.png}}
	\caption{Nonlinear Geometric Control Loop of the QR-Load system \label{fig:con.loop}}
\end{figure}	

The design of the controllers for the \a{qr} attitude can be found in \cite{Lee2010} and for the load attitude- and position this can be found in \cite{Sreenath2013c}. Thorough stability analyses are presented in either references. For a deeper understanding of Lyapunov stability analysis in geometric control, the reader can refer to \cite{Bullo2005}.

\paragraph{Configuration Errors}\label{sec:con.errors}
%CHECK which one is best?
%Most of nonlinear dynamics and control problems are studied in a linear space.\\
%\begin{equation}\label{key}
%\dot{x}=\textbf{f}(t,x,u), \quad x\in\mathbb{R}^n, u\in\mathbb{R}^m, \textbf{f}:\mathbb{R}^{n+m+1}\rightarrow\mathbb{R}^n
%\end{equation}
%OR
%In control systems engineering, the underlying geometric features of a dynamic system are often not considered carefully. For example, many control systems are developed for the standard form of ordinary differential equations, namely $ \dot{x}=f(x,u) $, where the state and the control input are denoted by $ x $ and $ u $. It is assumed that the state and the control input lie in Euclidean spaces, and the system equations are defined in terms of smooth functions between Euclidean spaces. However, for many interesting mechanical systems, the configuration space cannot be expressed globally as a Euclidean space.
The system dynamics evolve on nonlinear manifolds, that describe the configuration spaces for the \a{qr} attitude $ \in SO(3) $ and the load attitude $ \in \mathbb{S}^2 $. Likewise, configuration errors can be described on these manifolds. The derivation of the attitude and velocity errors can be found in \cite{Bullo2005}.
 
%CHECK 
%\cite{Maithripala2006}


The attitude error is denoted as $ R^T_dR $, and it describes the relative rotation from the body frame to the desired frame. 
The \a{qr} attitude error function on $ SO(3) $ is chosen to be 
\begin{equation}\label{eq:psiR}
\Psi_R(R,R_d)=\frac{1}{2}tr\left[I-R_d^TR\right]
\end{equation}
$ \Psi_R $ is locally positive-definite about $ R^T_dR=I $ within the region where the rotation angle between $ R $ and $ R_d $ is less than $ 180^\circ $. 
It can be shown that this region where $ \Psi_R<2 $ almost covers $ SO(3) $.
%ADD explain
% instead of comparing all elements of rotation matrix. PsiR is a measure for the error
%ADD 
%(physical) Meaning of the error functions
The derivative of Equation \ref{eq:psiR} is given by 
\begin{equation}\label{key}
\mathbf{D}_R\Psi(R,R_d)\cdot R\hat{\eta}=\frac{1}{2}(R^T_dR-R^TR_d)^\vee\cdot\eta
\end{equation}
where the \textit{vee map} $ ^\vee:\mathfrak{so}(3)\rightarrow\mathbb{R}^3 $ is the inverse of the \textit{hat map} defined in Section \ref{sec:mod.geometric}. From this, the attitude tracking error is chosen to be
\begin{equation}\label{key}
e_R=\frac{1}{2}(R_d^TR-R^TR_d)^\vee
\end{equation}
%CHECK
%The tracking error functions on $ TSO(3) $, the tangent space of $ SO(3) $, are defined as
%The attitude and angular velocity tracking error should be carefully chosen as they evolve on the tangent bundle of  $ SO(3) $. \cite{Lee2010c} 
The tangent vectors $ \dot{R} $ and $ \dot{R}_d $ cannot be compared directly, since they do not lie in the same space. They each are defined in their own tangent spaces, denoted by $ \dot{R} \in T_RSO(3)$ and $ \dot{R}_d \in T_{R_d}SO(3)$. 
%In order to define an error function, 
$ \dot{R}_d $ is transformed into a vector on $ T_RSO(3) $ to compare it with $ \dot{R} $. This can be done by an mathematical object called a \textit{transport map}, that enables the comparison of tangent vectors living in different spaces. 
%it connects tangent spaces at different points.
%CHeck Bullo p536
%CHeck Bullo p555

The velocity error that corresponds to the transport map is defined as
\begin{equation}\label{key}
\dot{e}=\dot{R}-\dot{R}_d(R_d^TR) 
\end{equation} 
Substituting Equations \ref{eq:SO3} and \ref{eq:Rdot}, $ \dot{R}_d $ can now be compared with $ \dot{R} $ through
\begin{equation}\label{key}
\begin{aligned}
\dot{R}-\dot{R}_d(R_d^TR) &=R\hat{\Omega}-R_d\hat{\Omega}_d(R_d^TR) \\
&=R(\Omega)^\wedge-(RR^T)R_d\hat{\Omega}_dR_d^TR\\
&=R(\Omega)^\wedge-R(R^TR_d{\Omega}_d)^\wedge \\
&=R(\Omega-R^TR_d{\Omega}_d)^\wedge 
\end{aligned}
\end{equation}


%CHECK wat wordt hier nou precies mee gezegd?
%With the use of Equation \ref{eq:Rdot} and \ref{eq:so3} 
%\begin{equation}\label{eq:dRd'R}
%\begin{aligned}
%\frac{d}{dt}(R^T_dR)&=(R^T_d\dot{R})+(\dot{R}^T_dR)\\
%&=(R^T_d(R\hat{\Omega}))+((R_d\hat{\Omega}_d)^TR)\\
%&=R^T_dR(\Omega)^\wedge-(\hat{\Omega}_dR_d^T)R\\
%&=R^T_dR(\Omega)^\wedge-(R^T_dRR^TR_d)\hat{\Omega}_dR_d^TR\\
%&=R^T_dR(\Omega-R^TR_d\Omega)^\wedge\\
%&=(R^T_dR)\hat{e}_\Omega
%\end{aligned}
%\end{equation}
%
%\begin{equation}\label{eq:ehatOmega}
%\hat{e}_\Omega = (\Omega-R^TR_d\Omega)^\wedge
%\end{equation}
%The tracking error for the angular velocity of the rotation matrix $ (R_d^TR)$ 
The velocity tracking error in \BF  can now be defined as 
\begin{align}\label{key}
e_\Omega&=\Omega- R^TR_d\Omega_d
\end{align}

The load attitude error function on $ \mathbb{S}^2 $ is chosen to be 
\begin{equation}\label{eq:psiq}
\Psi_q=1-q_d^Tq
\end{equation}

In the same fashion a \textit{transport map} is used for a comparison between the tangent spaces $ T_q\mathbb{S}^2$ and $ T_{q_d}\mathbb{S}^2$. This results in the following error functions on $ T\mathbb{S}^2 $
\begin{align}
e_q&=\hat{q}^2q_d\label{eq:con.eq}\\
e_{\dot{q}}&=\dot{q}-(q_d\times\dot{q}_d)\times q\label{eq:con.edq}
\end{align}

The tracking errors for position and velocity are defined as
\begin{align}\label{key}
e_x&=x-x_d\\
e_v&=v-v_d
\end{align}
where $ v_d=\dot{x}_d $ and $ x_d(t) \in \mathbb{R}^3$ is a smooth desired load position.

\section{Quadrotor Attitude Tracking}\label{sec:con.qratt}
QR Attitude Controlled Mode is designed to control the \a{qr} attitude by tracking a desired \a{qr} attitude command $ R_d(t) $ and a heading direction $ b_{1_d}(t) $. 

%\begin{figure}[h!]
%	\centering
%	\makebox[\textwidth][c]{\includegraphics[width=.45\textwidth]{./StyleStuff/dcsc.png}}
%	\caption{Quadrotor Attitude Controller\label{fig:con.qrattloop}}
%\end{figure}		

The calculation of the moment consists of a proportional term, a derivative term and a canceling term, and is defined as follows \cite{Lee2010}
\begin{equation}\label{eq:con.M}
M = \frac{1}{\epsilon^2}k_Re_R-\frac{1}{\epsilon}k_\Omega e_\Omega+\Omega\times J\Omega-J(\hat{\Omega}R^TR_d\Omega_d-R^TR_d\dot{\Omega}_d)
\end{equation}
%ADD Uitleggen waar dit vandaan komt?
for any positive constants $ k_R, k_\Omega $, and $ 0<\epsilon<1 $. Where \lsymb{$ \epsilon $}{Tuning parameter to enable rapid exponential convergence of $ e_R, e_\Omega $} is a parameter to enable rapid exponential convergence of the attitude error functions. $ \dot{\Omega}_d $ follows from 
\begin{equation}\label{key}
\begin{aligned}
\dot{R}_d&=R_d\hat{\Omega}_d\\
\hat{\Omega}_d&=R_d^T\dot{R}_d
\end{aligned}
\end{equation}
\begin{equation}\label{key}
\begin{aligned}
\dot{\hat{\Omega}}_d&=(\dot{R}_d^T\dot{R}_d)+(R_d^T\ddot{R}_d)\\
&=(R_d\hat{\Omega}_d)^T(R_d\hat{\Omega}_d)+(R_d^T\ddot{R}_d)\\
&=-\hat{\Omega}_d\hat{\Omega}_d+R_d^T\ddot{R}_d,\\
\hat{\Omega}_d&=(-\hat{\Omega}_d\hat{\Omega}_d+R_d^T\ddot{R}_d)^\vee
\end{aligned}
\end{equation}

%ADD explain that error function will be asymptoticallly stable for right parameters. larger region of attraction
It is proven in \cite{Lee2010} that the zero equilibrium of the closed loop tracking error $ (e_R,e_\Omega)=(0,0) $ is exponentially stable, if the initial conditions satisfy
\begin{equation}\label{eq:dom1}
\Psi_R(R(0),R_d(0))<2
\end{equation}
\begin{equation}\label{eq:dom2}
\parallel e_\Omega(0)\parallel^2<\frac{2}{\lambda_M(J)}\frac{k_R}{\epsilon^2}(2-\Psi_R(R(0),R_d(0)))
\end{equation}
where \lsymb{$ \lambda_M(\cdot) $}{Maximum eigenvalue} denotes the maximum eigenvalue.

Furthermore, there exist constants $ \alpha_R,\beta_R>0 $ such that
\begin{equation}\label{eq:con.PsiRconv}
\Psi_R(R(t),R_d(t)) \leq min\left\lbrace 2,\alpha_Re^{-\beta_Rt}\right\rbrace 
\end{equation}

The domain of attraction is defined by Equations \ref{eq:dom1} and \ref{eq:dom2}. \cite{Lee2010} shows the derivation of a stability analysis of the controller is presented. 

%CHECK what this is about
%Asymptotic tracking of the quadrotor attitude does not require specification of the thrust magnitude. As an auxiliary problem, the thrust magnitude can be chosen in many different ways to achieve an additional translational motion objective. For example, it can be used to asymptotically track a quadrotor altitude command [28]. Since the translational motion of the quadrotor UAV can only be partially controlled; this flight mode is most suitable for short time periods where an attitude maneuver is to be completed. \cite{Goodarzi2015b}

%CHECK  waar zijn de feedforward termen?
%CHECK komt dit voort uit "Exact linearization" of "dynamic inversion"? 
%Control input \cite{Lee2011}
%\begin{equation}\label{eq:inputattitude}
%u=-k_Re_R-k_\Omega\Omega-mg\rho\times R^Te_3
%\end{equation}
%Insert into Equation \ref{eq:eomrigidbody}; closed loop dynamics are given by
%\begin{align}\label{eq:CLdynamics}
%J\dot{\Omega} &= -\Omega\times J\Omega-k_Re_R-k_\Omega\Omega \\
%\dot{R} &= R\hat{\Omega}
%\end{align}

%ADD stability analysis


%CHECK uberhaupt wel nodig?
%The error dynamics for $ e_R $ is defined as
%\begin{equation}\label{key}
%\begin{aligned}
%\dot{e}_R&=\frac{1}{2}(R_d^TR\hat{e}_\Omega+\hat{e}_\Omega R^TR_d)^\vee\\
%&=\frac{1}{2}(tr[R^TR_d]I-R^TR_d)\hat{e}_\Omega\equiv C(R^T_dR)e_\Omega
%\end{aligned}
%\end{equation}
%The error dynamics for $ e_\Omega$ is obtained by substituting Equation \ref{eq:ehatOmega} into Equation \ref{eq:con.M}
%\begin{equation}\label{key}
%\begin{aligned}
%\dot{e}_\Omega&=J^{-1}(-k_Re_R-k_\Omega e_\Omega)
%\end{aligned}
%\end{equation}

\section{Load Attitude Tracking}\label{sec:con.loadatt}

%ADD How is the controller built.

%ADD Dependent of what values? 	How to choose parameters.

%\begin{figure}[h!]
%	\centering
%	\makebox[\textwidth][c]{\includegraphics[width=.45\textwidth]{./StyleStuff/dcsc.png}}
%	\caption{\label{fig:con.loadattloop}}
%\end{figure}		

The Load Attitude Controlled Mode tracks a desired load attitude $ q_d $ by calculating a command signal for the \a{qr} attitude, defined as
\begin{equation}\label{eq:con.R}
R_c = \begin{bmatrix}
b_{1c}; b_{3c}\times b_{1c};b_{3c}
\end{bmatrix}
\end{equation}
where $ b_{3c} \in \mathbb{S}^2 $ is defined by 
\begin{equation}\label{eq:con.b3c}
b_{3c}=\frac{F}{||F||}
\end{equation}
Such that $ F $ in Equation \ref{eq:con.b3c} is defined by a normal component $ F_n $, $ F_{pd} $ and $ F_{ff}$
\begin{equation}\label{key}
F=F_n-F_{pd}-F_{ff}
\end{equation}
 Control forces for a system evolving on $ \mathbb{S}^2 $, are derived in \cite{Bullo2005}. 
 This results in a proportional-derivative force $ F_{pd} $ and a feed forward force $ F_{ff} $, that are functions of Equations \ref{eq:con.eq} and \ref{eq:con.edq}. The following terms are obtained
\begin{equation}\label{key}
\begin{aligned}
F_{pd}&=-k_P\hat{q}^2q_d-k_D(\dot{q}-(q_d\times\dot{q}_d\times q)\\
&=-k_qe_q-k_\omega e_{\dot{q}}
\end{aligned}
\end{equation}
\begin{equation}\label{key}
F_{ff}=m_QL\langle\langle q,q_d\times\dot{q}_d\rangle\rangle_{\mathbb{R}^3}(q\times \dot{q})+m_QL(q_d\times \ddot{q}_d)\times q
\end{equation}
The unit vector $ b_{1c} $ is defined as
\begin{equation}\label{key}
b_{1c}=-\frac{1}{||b_{3c}\times b_{1d}||}(b_{3c}\times(b_{3c}\times b_{1d}))
\end{equation}
where $ b_{1d}\in \mathbb{S}^2 $ is chosen, not parallel to $ b_{3c} $.
The total upward thrust is defined as
\begin{equation}\label{key}
f=F\cdot Re_3
\end{equation}

%ADD explain that error function will be asymptoticallly stable for right parameters. larger region of attraction
It is proven in \cite{Sreenath2013c} that the zero equilibrium of the closed loop tracking error $ (e_q,e_{\dot{q}},e_R,e_\Omega)=(0,0,0,0) $ is exponentially stable, if the initial conditions satisfy
\begin{equation}\label{eq:dom3}
\Psi_q(q(0),q_d(0))<2
\end{equation}
\begin{equation}\label{eq:dom4}
\parallel e_{\dot{q}}(0)\parallel^2<\frac{2}{m_QL}{k_R}(2-\Psi_q(q(0),q_d(0)))
\end{equation}

The domain of attraction is defined by Equations \ref{eq:dom1}, \ref{eq:dom2}, \ref{eq:dom3} and \ref{eq:dom4}.
Furthermore, there exist constants $ \alpha_q,\beta_q>0 $ such that
\begin{equation}\label{eq:con.Psiqconv}
\Psi_q(q(t),q_d(t)) \leq min\left\lbrace 2,\alpha_qe^{-\beta_qt}\right\rbrace 
\end{equation}

\section{Load Position Tracking}\label{sec:con.loadpos}

%CHECK nodig?
%Explain how $ f $ and $ \vec{b_{1_d}} $ is obtained from $ x_d(t) $?\\

Tracks load position reference. Outputs load attitude reference.

\begin{equation}\label{eq:con.q}
q_c = - \frac{A}{||A||}
\end{equation}
where
\begin{equation}\label{key}
A = -k_xe_x-k_ve_v+(m_Q+m_L)(\ddot{x}_{L,d}+ge_3)+m_QL(\dot{q}\cdot\dot{q})q
\end{equation}
with $ e_x=x_L-x_{L,d} $ and $ e_v=\dot{x}_L-\dot{x}_{L,d} $.
Furthermore, $ F_n $ is redefined as
\begin{equation}\label{key}
F_n=(A\cdot q)q
\end{equation}
%\begin{figure}[h!]
%	\centering
%	\makebox[\textwidth][c]{\includegraphics[width=.45\textwidth]{./StyleStuff/dcsc.png}}
%	\caption{\label{fig:con.loadposloop}}
%\end{figure}		

%CHECK error function will be asymptoticallly stable for right parameters? 
It is proven in \cite{Sreenath2013c} that the zero equilibrium of the closed loop tracking error $ (e_x,e_v,e_q,e_{\dot{q}},e_R,e_\Omega)=(0,0,0,0,0,0) $ is exponentially stable, if the initial conditions satisfy
\begin{equation}\label{eq:dom5}
\Psi_q(q(0),q_c(0))<\psi_1<1
\end{equation}
\begin{equation}
\parallel e_{x}(0)\parallel^2<e_{x_{max}}
\end{equation}
where $ e_{x_{max}} $ and $ \psi_1 $ are fixed constants. 

The domain of attraction is defined by Equations \ref{eq:dom1}, \ref{eq:dom2}, \ref{eq:dom5} and the following equation
\begin{equation}
\parallel e_{\dot{q}}(0)\parallel^2<\frac{2}{m_QL}{k_q}(\psi_1-\Psi_q(q(0),q_d(0)))
\end{equation}


Furthermore, there exist constants $ \alpha_q,\beta_q>0 $ such that
\begin{equation}\label{key}
\Psi_q(q(t),q_d(t)) \leq min\left\lbrace 2,\alpha_qe^{-\beta_qt}\right\rbrace 
\end{equation}

\section{Stability Analysis}\label{sec:con.sta}
Lyapunov Analysis on SO3 x R3 and S2 x R3
Closed-loop full-attitude dynamics evolve on the non- Euclidean manifold SO3 x R3. 
Since these manifolds are locally Euclidean, local stability properties of a closed-loop equilibrium solution can be assessed using standard Lyapunov methods. 
In addition, the LaSalle invariance result and related Lyapunov results apply to closed-loop vector fields defined on these manifolds. 
However, since the manifolds SO3 and S2 are compact, the radial unboundedness assumption cannot be satisfied; 
consequently, global asymptotic stability cannot follow from a Lyapunov analysis on Euclidean spaces [40], and therefore must be analyzed in alternative ways [19]–[23].\cite[p.43]{Chaturvedi2011}

\cite{Chaturvedi2011} summarizes global results on attitude control and stabilization for a rigid body using continuous time- invariant feedback. The analysis uses methods of geometric mechanics based on the geometry of the special orthogonal group SO3 and the two-sphere S2.



%HOeft waarschijnlijk niet eens een section te zijn, kan kort en bondig
\section*{Summary}

%ADD What is Geometric Control? Differences between other control
Control design is based on Nonlinear Geometric Control.

%ADD Why Geometric Control? Why is useful
%PRO
%The proposed control system is robust to switching conditions since each flight mode has almost global stability properties, and it is straightforward to design a complex maneuver of a QR. \cite{Lee2010c}




