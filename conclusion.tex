\chapter{Conclusions and Future Work}\label{ch:conclusion}
%ADD in this section etc

\section{Summary and Conclusions}
The report starts with an introduction to the subject.
The aim is described and the motivation for this research is given.

In the following chapter
A short introduction of the concepts of differential geometry

Next, the nonlinear geometric control design is explained.
Backstepping is explained

Experiments are defined.
Testing the nonlinear Geometric Controller
To compare with a common linear controller, \a{lqr} control is used to compare
Results are,

The conclusions we could extract from the experiments is that

\section{Recommendations for Future Work}\label{ch:future}
% DONT make it look like literature survey. Based on details, show what I have learned. What could be interesting next. What is missing for the next steps.

\subsection{Investigate Implementation}
Lack of time: no implementation possible. In order to realize this, one needs to investigate the subjects described in this section.

Can in be implemented in terms of computational power? Can this be run on on-board processor?
Problem is already partly simplified by command filter. Quantification and comparison between implemented \a{mpc} and Geometric Control.

Model identification and validation. Now it is an estimation and/or taken from other literature.

How to deal with noise?

%CHECK wat is dit precies?
%Computational Geometric Mechanics and Control\\
%\cite{Lee2008}
%Computation algorithms must be developed which preserve the geometric properties of a mechanical system.\\
%Robust and careful numerical implementation of geometric control theory to complex engineering systems.\\
%Provides nontrivial maneuvers that are globally valid on a nonlinear configuration manifold.


\subsection{Modeling Constraints}
There are several techniques to handle input saturation, the most popular ones are anti-windup techniques. Back-calculation is such a method for PID to activate the integrator, is this possible for NL control?

***************************************\\
\cite{Goodarzi2013a} includes uncertainties in the translational dynamics and rotational dynamics. Out of the scope, might be interesting.\\
***************************************\\

\subsection{Hybrid Modeling}
Switching between several flight modes yields autonomous acrobatic maneuvers. Robust to switching conditions ***why?\\
\cite{Tang2014}

\subsection{Trajectory Generation}
\subsubsection{Minimum Snap Trajectory Generation}

Trajectory can be generated by solving a \a{QP} via minimum snap generation.

Problem in smaller 4-D space instead of 12-D, with help of differential flatness. Explain differential flatness and its usefulness.

Is able to include constraints in \a{QP}.

\cite{Mellinger2011}

Obstacle avoidance

