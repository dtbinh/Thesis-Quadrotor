\chapter{Results}\label{ch:results}
The sections in this chapter discuss the results that are obtained from the load trajectory tracking experiments.

In Figures \ref{fig:set.caseAres}, \ref{fig:set.caseBres} and \ref{fig:set.caseCres} the load tracking performance is shown for the Nonlinear Geometric Controller. From these figures the tracking errors of the \a{qr} attitude, load attitude and load position, and the stability of the tracking errors can be analyzed.

\newpage
\section{Case A}
Figure \ref{fig:AxLlqr} shows the load position along the desired load position $ x_{L,d} $ for both control approaches.\\
Figure \ref{fig:AexLlqr} shows the load position error for both control approaches.

Figure \ref{fig:AQRang} shows the \a{qr} attitude with respect to \IF.\\
In Figure \ref{fig:ALang} the load angle with respect to \BF is shown.\\
Observations: 

\begin{figure}[h!]
	\centering
	\makebox[.49\textwidth][c]{\subfloat[][Load tracking \label{fig:AxLlqr}]{\includegraphics[width=.525\textwidth]{\dir{LQR-xL40}}}}
	\makebox[.49\textwidth][c]{\subfloat[][Load Position Error\label{fig:AexLlqr}]{\includegraphics[width=.525\textwidth]{\dir{LQR-exL40}}}}
	\makebox[.49\textwidth][c]{\subfloat[][QR Attitude\label{fig:AQRang}]{\includegraphics[width=.525\textwidth]{\dir{LQR-QRang40}}}}
	\makebox[.49\textwidth][c]{\subfloat[][Load Attitude\label{fig:ALang}]{\includegraphics[width=.525\textwidth]{\dir{LQR-Lang40}}}}
	\caption{Controller Comparison Case A\label{fig:}}
\end{figure}	

The desired and actual load trajectory, and the position error are shown in Figure \ref{fig:AxL} and Figure \ref{fig:AexL}, respectively.
From this can be seen that a small steady state error remains in the z-direction. However, $ (e_x,e_v)=(0,0) $ is exponentially attractive. 
%CHECK kan deze error nog verholpen worden? Gain ex ev omhoog? 

Figure \ref{fig:AeR} and \ref{fig:Aeq} show the tracking errors of the \a{qr} attitude and load attitude, respectively.\\
Observations: $(e_x,e_v,e_q,e_{\dot{q}},e_R,e_\Omega)=(0,0,0,0,0,0) $ is exponentially stable

Figure \ref{fig:APsiR} and \ref{fig:APsiq} show the tracking error functions of the \a{qr} and load, respectively. \\
Observations: there exist constants $ \alpha_q,\beta_q>0 $ such that
\begin{equation}\label{key}
\Psi_q(q(t),q_d(t)) \leq min\left\lbrace 2,\alpha_qe^{-\beta_qt}\right\rbrace 
\end{equation}

%Figure \ref{fig:AeR} and Figure \ref{fig:Aeq} also show that the tracking error for both attitude and angular velocity are exponentially attractive.
%Figure \ref{fig:APsiR} confirms Equation \ref{eq:con.PsiRconv} and likewise, Figure \ref{fig:APsiq} confirms Equation \ref{eq:con.Psiqconv} and it can be seen that both the configuration error $ \Psi_R $ on $ SO(3) $ and $ \Psi_q$ on $ \mathbb{S}^2 $ is smaller than 2 and converges to zero along the trajectory.

%
%\begin{figure}[h!]
%	\centering
%	\makebox[\textwidth][c]{\includegraphics[width=.75\textwidth]{\dir{LQR-exL40}}}
%	\caption{ \label{fig:set.caseAexL}}
%\end{figure}		

\begin{figure}[h!]
	\centering
	\makebox[.49\textwidth][c]{\subfloat[][]{\includegraphics[width=.5\textwidth]{\dir{LPOSQRL-xL40}}\label{fig:AxL}}}	
	\makebox[.49\textwidth][c]{\subfloat[][]{\includegraphics[width=.5\textwidth]{\dir{LPOSQRL-exL40}}\label{fig:AexL}}}	
	\makebox[.49\textwidth][c]{\subfloat[][]{\includegraphics[width=.5\textwidth]{\dir{LPOSQRL-eR40}}\label{fig:AeR}}}
	\makebox[.49\textwidth][c]{\subfloat[][]{\includegraphics[width=.5\textwidth]{\dir{LPOSQRL-eq40}}\label{fig:Aeq}}}
	\makebox[.49\textwidth][c]{\subfloat[][]{\includegraphics[width=.5\textwidth]{\dir{LPOSQRL-PsiR40}}\label{fig:APsiR}}}
	\makebox[.49\textwidth][c]{\subfloat[][]{\includegraphics[width=.5\textwidth]{\dir{LPOSQRL-Psiq40}}\label{fig:APsiq}}}
	\caption{Results Nonlinear Geometric Control Case A \label{fig:set.caseAres}}
\end{figure}	


\newpage
\section{Case B}
Figure \ref{fig:BxLlqr} shows the load position along the desired load position $ x_{L,d} $ of both controllers.

\begin{figure}[h!]
	\centering
	\makebox[.49\textwidth][c]{\subfloat[][Load tracking \label{fig:BxLlqr}]{\includegraphics[width=.525\textwidth]{\dir{dcsc}}}}
	\makebox[.49\textwidth][c]{\subfloat[][Load Position Error\label{fig:BexLlqr}]{\includegraphics[width=.525\textwidth]{\dir{dcsc}}}}
	\caption{Controller Comparison Case B\label{fig:}}
\end{figure}		


\newpage
\begin{figure}[h!]
	\centering
%\makebox[.49\textwidth][c]{\subfloat[][]{\includegraphics[width=.5\textwidth]{\dir{LPOSQRL-xLB}}\label{fig:}}}	
%\makebox[.49\textwidth][c]{\subfloat[][]{\includegraphics[width=.5\textwidth]{\dir{LPOSQRL-exLB}}\label{fig:}}}	
%\makebox[.49\textwidth][c]{\subfloat[][]{\includegraphics[width=.5\textwidth]{\dir{LPOSQRL-eqB}}\label{fig:}}}
%\makebox[.49\textwidth][c]{\subfloat[][]{\includegraphics[width=.5\textwidth]{\dir{LPOSQRL-eRB}}\label{fig:}}}
%\makebox[.49\textwidth][c]{\subfloat[][]{\includegraphics[width=.5\textwidth]{\dir{LPOSQRL-PsiRB}}\label{fig:}}}
%\makebox[.49\textwidth][c]{\subfloat[][]{\includegraphics[width=.5\textwidth]{\dir{LPOSQRL-PsiqB}}\label{fig:}}}
\makebox[.49\textwidth][c]{\subfloat[][]{\includegraphics[width=.5\textwidth]{\dir{dcsc}}\label{fig:}}}	
\makebox[.49\textwidth][c]{\subfloat[][]{\includegraphics[width=.5\textwidth]{\dir{dcsc}}\label{fig:}}}	
\makebox[.49\textwidth][c]{\subfloat[][]{\includegraphics[width=.5\textwidth]{\dir{dcsc}}\label{fig:}}}
\makebox[.49\textwidth][c]{\subfloat[][]{\includegraphics[width=.5\textwidth]{\dir{dcsc}}\label{fig:}}}
\makebox[.49\textwidth][c]{\subfloat[][]{\includegraphics[width=.5\textwidth]{\dir{dcsc}}\label{fig:}}}
\makebox[.49\textwidth][c]{\subfloat[][]{\includegraphics[width=.5\textwidth]{\dir{dcsc}}\label{fig:}}}
	\caption{Results Nonlinear Geometric Control Case B \label{fig:set.caseBres}}
\end{figure}		


\newpage
\section{Case C}

Figure \ref{fig:CxLlqr} shows the load position along the desired load position $ x_{L,d} $ of both controllers.\\
Figure \ref{fig:CexLlqr} shows the load position error for both control approaches.\\
Observations: fact that LQR can not control load position is obvious.\\
OTHER GAINS FOR LQR!\\
Very small penalty on load angle results in swinging load; decreasing load position error, but very bad anti-swing. 

Figure \ref{fig:CQRang} shows the \a{qr} attitude with respect to \IF.\\
In Figure \ref{fig:CLang} the load angle with respect to \BF is shown. \\
Observations: Load angles are huge, check results

\begin{figure}[h!]
	\centering
	\makebox[.49\textwidth][c]{\subfloat[][Load tracking \label{fig:CxLlqr}]{\includegraphics[width=.525\textwidth]{\dir{LQR-xL41}}}}
	\makebox[.49\textwidth][c]{\subfloat[][Load Position Error\label{fig:CexLlqr}]{\includegraphics[width=.525\textwidth]{\dir{LQR-exL41}}}}
	\makebox[.49\textwidth][c]{\subfloat[][QR Attitude\label{fig:CQRang}]{\includegraphics[width=.525\textwidth]{\dir{LQR-QRang41}}}}
\makebox[.49\textwidth][c]{\subfloat[][Load Attitude\label{fig:CLang}]{\includegraphics[width=.525\textwidth]{\dir{LQR-Lang41}}}}	
	\caption{Controller Comparison Case C\label{fig:}}
\end{figure}	

While tracking the required \a{qr} attitude, which tilts the \a{qr} to reach the desired velocities in the right direction, it can be seen that the system has difficulties to also maintain the desired height, which can be explained by the fact that the total force will not point upwards if the \a{qr} is tilted. Despite the fact that the \a{qr} is  moving from side to side, the upward force is still controlled to track the desired height. 

Figure \ref{fig:CxL} shows the desired load position, and Figure \ref{fig:CexL} shows that the error is mainly the overshoot in the x-direction, due to the fast desired swinging motion. 

Figure \ref{fig:CeR} and \ref{fig:Ceq} show the tracking errors of the \a{qr} attitude and load attitude, respectively. \\
Observations: $(e_x,e_v,e_q,e_{\dot{q}},e_R,e_\Omega)=(0,0,0,0,0,0) $ is exponentially stable

Figure \ref{fig:CPsiR} and \ref{fig:CPsiq} show the tracking error functions of the \a{qr} and load, respectively. \\
Observations: there exist constants $ \alpha_q,\beta_q>0 $ such that
\begin{equation}\label{key}
\Psi_q(q(t),q_d(t)) \leq min\left\lbrace 2,\alpha_qe^{-\beta_qt}\right\rbrace 
\end{equation}
	

%CHECK right pictures?
\begin{figure}[h!]
	\centering
\makebox[.49\textwidth][c]{\subfloat[][]{\includegraphics[width=.5\textwidth]{\dir{LPOSQRL-xL41}}\label{fig:CxL}}}	
\makebox[.49\textwidth][c]{\subfloat[][]{\includegraphics[width=.5\textwidth]{\dir{LPOSQRL-exL41}}\label{fig:CexL}}}	
\makebox[.49\textwidth][c]{\subfloat[][]{\includegraphics[width=.5\textwidth]{\dir{LPOSQRL-eR41}}\label{fig:CeR}}}
\makebox[.49\textwidth][c]{\subfloat[][]{\includegraphics[width=.5\textwidth]{\dir{LPOSQRL-eq41}}\label{fig:Ceq}}}
\makebox[.49\textwidth][c]{\subfloat[][]{\includegraphics[width=.5\textwidth]{\dir{LPOSQRL-PsiR41}}\label{fig:CPsiR}}}
\makebox[.49\textwidth][c]{\subfloat[][]{\includegraphics[width=.5\textwidth]{\dir{LPOSQRL-Psiq41}}\label{fig:CPsiq}}}
	\caption{Results Nonlinear Geometric Control Case C \label{fig:set.caseCres}}
\end{figure}	

\newpage
\section{Conclusion}\label{set:set.con}
%CHECK 
%What can we learn and conclude from different performance comparisons
%CHECK
%What is its value of nonlinear control compared to linear control

Near the equilibrium configuration, the \a{lqr} controller is able to reduce the swing. In fast trajectories however, the shortcomings of the \a{lqr} controller become evident. 

The nonlinear geometric controller depends on feed forward terms that are obtained from the desired trajectories. 
Trajectory generation approaches exist that are able to generate the required desired position, velocity and acceleration by 
however it is possible to compute these with trajectory generating algorithms too.

The controllers are functions of the computed tracking references $ q_c, R_c $ and their derivatives. These terms are approximated by a command filter, which means that the accuracy decreases because high frequency terms are filtered.

%ADD conclusie 
% over parameter keuze in controllers. Arbitrair, maar wellicht later wel van belang